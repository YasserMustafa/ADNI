\documentclass[12pt,a4paper]{article}
\usepackage{mathtools}
\usepackage{parskip}
\usepackage[colorlinks=true, urlcolor=blue]{hyperref}
\usepackage{amsmath}
\usepackage{amssymb}
\usepackage{float}
\usepackage{fontenc}
\usepackage{graphicx}
\usepackage{caption}
\usepackage{subcaption}
\usepackage{multirow}

\renewcommand\thesubsection{\thesection.\arabic{subsection}}
\renewcommand\thesubsubsection{\thesubsection.\alph{subsubsection}}

\newcommand{\tab}{\hspace*{2em}}

\title{ADNI Progress report}
\author{Devendra Goyal\\Uniqname: devendra}

\date{\today}

\begin{document}
\maketitle

\part{Using HMMs to predict disease progression}

Training set: randomly picked 1000 patients.\\
Testing set: Remaining 398 patients.

\section{Sanity Check}
\label{sec:sanity}

As a sanity check, the HMM was trained with the number of states
$k=3$.

Following are the temporal parameters, as learned by the HMM and
contrasted with the ground truth.

\[
gt.pi = \left[ 
\begin{array}{ccc} 
0.3062 & 0.4667 & 0.2272
\end{array}
 \right]
\]
\[
model.pi = \left[ 
\begin{array}{ccc} 
0.3352 & 0.1826 & 0.4822
\end{array}
 \right]
\]
\[
gt.A = \left[ 
\begin{array}{ccc} 
0.9359 & 0.0569 & 0.0071 \\
0.0260 & 0.8516 & 0.1224 \\
0 & 0.0084 & 0.9916 \\
\end{array}
 \right]
\]
\[
model.A = \left[ 
\begin{array}{ccc} 
0.9721 & 0.0258 & 0.0022 \\
0.0040 & 0.9923 & 0.0037 \\
0.0090 & 0.0146 & 0.9764 \\
\end{array}
 \right]
\]

\begin{figure}[H]
  \centering
  \includegraphics[width=\textwidth]{hmm/sanity-train.png}
  \caption{Viterbi trellis for training set}  
\end{figure}

\begin{figure}[H]
  \includegraphics[width=\textwidth]{hmm/sanity-test.png}
  \caption{Viterbi trellis for testing set}  
\end{figure}

\begin{figure}[H]
  \centering
  \includegraphics[width=\textwidth]{hmm/sanity-viterbi-train.png}
  \caption{Distribution of states for training set}  
\end{figure}

\begin{figure}[H]
  \includegraphics[width=\textwidth]{hmm/sanity-viterbi-test.png}
  \caption{Distribution of states for testing set}  
\end{figure}

\subsection{Analysing the transitions}
\label{sec:an-trans}

The following is a matrix that visualizes where MCI$\rightarrow$AD
transitions in the labelled data end up in the HMM state space. That
is, for every MCI$\rightarrow$AD transition in the real world, we look
at the corresponding transition in the HMM state space. The rows
represent the source HMM state, whereas the columns are the
destination HMM state. This data is aggregated over time, so there is
no temporal aspect. 

For the training set, the conversion matrix is:

\[
\left[ 
\begin{array}{ccc} 
37 & 2 & 0 \\
0  & 36 & 0 \\
0 & 2 & 16 \\
\end{array}
 \right]
\]

For the testing set, the conversion matrix is:

\[
\left[ 
\begin{array}{ccc} 
13 & 0 & 0 \\
0  & 6 & 0 \\
0 & 0 & 1 \\
\end{array}
 \right]
\]

That is, MCI$\rightarrow$AD transitions in the real world are
reflected as stationary transitions in the HMM state space.

\section{HMM with larger state space}

\[
model.pi = \left[ 
\begin{array}{ccccc} 
  0.2103 & 0.2825 & 0.1518 & 0.1325 & 0.2229\\
\end{array}
 \right]
\]
\[
model.A = \left[ 
\begin{array}{ccccc} 
0.9424 & 0.0043 & 0.0426 & 0.0061 & 0.0046 \\
0.0224 & 0.9445 & 0.0055 & 0.0099 & 0.0177 \\
0.0040 & 0.0036 & 0.9842 & 0.0046 & 0.0037 \\
0.0066 & 0.0062 & 0.0073 & 0.9720 & 0.0079 \\
0.0046 & 0.0045 & 0.0114 & 0.0063 & 0.9733 \\
\end{array}
 \right]
\]

\begin{figure}[H]
  \centering
  \includegraphics[width=\textwidth]{hmm/trellis-5-train.png}
  \caption{Viterbi trellis for training set}  
\end{figure}

\begin{figure}[H]
  \includegraphics[width=\textwidth]{hmm/trellis-5-test.png}
  \caption{Viterbi trellis for testing set}  
\end{figure}

\begin{figure}[H]
  \centering
  \includegraphics[width=\textwidth]{hmm/viterbi-5-train.png}
  \caption{Distribution of states for training set}  
\end{figure}

\begin{figure}[H]
  \includegraphics[width=\textwidth]{hmm/viterbi-5-test.png}
  \caption{Distribution of states for testing set}  
\end{figure}

\end{document}